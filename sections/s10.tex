\section{Потоковые алгоритмы на графах} % (fold)
\label{sec:graph_stream}

\paragraph{Графовые потоки} % (fold)
\label{par:graph_streams}

\begin{itemize}
	\item Незайтейлево получаем, что даже связность требует $\Omega(n)$ памяти.
	\item Цель сделать $O(space \slash n)$ как можно меньше.
	\item Поток имеет вид $\langle (u_1, v_1), \cdots, (u_m, v_m) \rangle$.
	\item Здесь может потребоваться рассказать про link-cut деревья.
\end{itemize}

% paragraph graph_streams (end)

\paragraph{Связность графа} % (fold)
\label{par:connectedness}

\begin{itemize}
	\item Строим МСТ через СНМ.
	\item Получаем счастье.
\end{itemize}

% paragraph connectedness (end)

\paragraph{Двудольность} % (fold)
\label{par:bipartite}

\begin{itemize}
	\item Строим МСТ через СНМ.
	\item Запоминаем расстояние до корня. Проверяем не получится ли как-нибудь нечетного цикла.
	\item * Монотонность: если граф однажды испортится, то дальше уже неважно, что будет.
\end{itemize}

% paragraph bipartite (end)

\paragraph{Расстояние} % (fold)
\label{par:distance}

\begin{itemize}
	\item Процессим ребро граф $G$. Если оно не лежит в цикле длины $< t$, то добавляем в граф.
	\item Построили граф $H$. $dist_G(u, v) \leq dist_H(u, v) \leq t \cdot dist_G(u, v)$.
	\item Пографы без циклов длины меньшей $t$ называются $t$-спаннерами.
	\item Теорема Баллоба. В таких графах $O(n^{1 + \frac{2}{t}})$ ребер.
	\item Можно убрать из графа все вершины со степенью меньшей $\frac{m}{n}$. Число оставшихся ребер будет хотя бы $n$.
	\item Запускаем BFS. В нем будет $\frac{t - 1}{2}$ ветвлений со степень хотя бы $\frac{m}{n}$. Получается, вершин не меньше, чем $\left(\frac{m}{n}\right)^{\frac{t - 1}{2}}$. Отсюда выводим $m \leq n^{1 + \frac{2}{t - 1}}$.
\end{itemize}

% paragraph distance (end)

% section graph_stream (end)