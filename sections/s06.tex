\section{Оценка частотных моментов} % (fold)
\label{sec:frequent_moment}

\paragraph{Задача} % (fold)
\label{par:problem}

\begin{itemize}
	\item Ванильная модель. Дан поток $\sigma = (a_1, a_2, \cdots, a_m)$, все $a_i \in [n]$. Имеем частотный вектор $f$ в неявном виде. $||f||_1 = m$. 
	\item Найти $F_k = \sum_i f_i^k$.
	\item Строим $(\epsilon, \delta)$-аппроксимацию.
\end{itemize}

\paragraph{Алгоритм AMS} % (fold)
\label{par:ams_algo}

\begin{itemize}
	\item Тыкаем случайный элемент. 
	\item Начинаем считать, сколько раз он встретится в потоке. Пусть это будет r.
	\item Возвращаем $m (r^k - (r - 1)^k)$.
\end{itemize}

% paragraph ams_algo (end)

\paragraph{Анализ} % (fold)
\label{par:analysis}

\begin{itemize}
	\item Сначала показываем, что мат. ожидание хорошее. В lecnotes сложновато написано, просто перебираем все элементы.
	\item 
	\item 
\end{itemize}

% paragraph analysis (end)

% section frequent_moment (end)