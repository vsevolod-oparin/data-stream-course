\section{Непересекаемость множеств и нижние оценки} % (fold)
\label{sec:sd_lb}

\paragraph{CC для DISJ} % (fold)
\label{par:cc_disj}

\begin{itemize}
	\item Сведение INDEX к DISJ. 
	\item Одно множество это позиции единичек в $x$, другое, одноэлементное, состоит из одной запрашиваемой позиции.
	\item Счастье.
\end{itemize}

% paragraph cc_disj (end)

\paragraph{Связность графа} % (fold)
\label{par:st_connectivity}

\begin{itemize}
	\item Сведим DISJ к связности.
	\item Заводим $n + 2$ веришины: $n$ для множества + $s$ + $t$. 
	\item $s$ соединяем с тем, что есть у Алисы; $t$ с тем, что у Боба.
	\item Связность $\Leftrightarrow$ пересекаемость.
\end{itemize}

% paragraph st_connectivity (end)

\paragraph{Совершенное паросочетание} % (fold)
\label{par:perfect_matching}

\begin{itemize}
	\item Сводим INDEX к паросочетанию.
	\item Делаем граф с $V = \cup_i \{a_i, b_i, u_i, v_i\}$.
	\item $E_A = \{ (u_i, v_j) \,|\, x_{f(i, j)} = 1 \}$.
	\item $E_B = \{(a_l, u_l) \,|\,l \neq i\} \cup \{(b_l, v_l \,|\, l \neq j) \} \cup \{(a_i, b_j)\,|\,f(i, j) = k\}$.
	\item Счастье.
\end{itemize}

% paragraph perfect_matching (end)

\paragraph{Норма $F_k$ и мультиплеерный DISJ} % (fold)
\label{par:f_k_disj}

\begin{itemize}
	\item Теорема $R(DISJ_{n, t}) \geq \Omega(n \slash t)$.
	\item Однопроходный алгоритм для $F_k$ требует $\Omega(m^{1 - 2 \slash k})$.
	\item Берем набор $t$ векторов $x_1, \cdots, x_t$. Если составить матрицу вида
	$$
	\left(
	\begin{array}{c}
		x_1 \\
		x_2 \\
		\cdots \\
		x_t \\
	\end{array}
	\right)
	$$
	\item В каждой колонке может быть $0$, $1$ или $t$ единичек. $t$ единичек только в одной колонке.
	\item Если DISJ = 0, то $F_k \leq n$, иначе $F_k \geq t^k$. Если взять $t$ такое, что $t^k \geq 2 \cdot m$,
	то 2-аппроксимация умеет различать эти два случая.
	\item Памяти используется $\Omega(m \slash t^2) = \Omega(m^{1 - 2 \slash k})$.
\end{itemize}

% paragraph f_k_disj (end)

% section sd_lb (end)